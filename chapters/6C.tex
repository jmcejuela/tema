\chapter{Bibliography}
\label{Bibliography}
$\big[1\big]$ Ngan Nguyen, Jin-Dong Kim and Jun'ichi Tsujii. Overview of the Protein Coreference task in BioNLP Shared Task 2011.\emph{In Proceedings of BioNLP Shared Task 2011 Workshop}, pages 74-82 Portland, Oregon, USA, 24 June, 2011.  2011 Association for Computational Linguistics\\  \\ 
\big[2\big] Xiaofeng Yang, Jian Su, and Chew Lim Tan. 2004a Improving noun phrase coreference resolution by matching strings. \emph{In Proceedings of the First International Joint Conference on Natural Language Processing}, pages 22-31.\\  \\
\big[3\big] Ngan Nguyen, Jin-Dong Kim, Makoto Miwa, Takuya Matsuzaki and Jun'ichi Tsujii. Improving protein coreference resolution by simple semantic classification. November 2012 BMC Bioinformatics, 13:304\\  \\
\big[4\big] Wee Meng Soon, Hwee Tou Ng, and Daniel Chung Yong Lim. A machine learning approach to coreference resolution of noun phrases 2001. \emph{Computational linguistics}, 27(4):521-544\\  \\
\big[5\big] Yannick Versley, Simone Paolo Ponzetto, Massimo Poesio, Vladimir Eidelman, Alan Jern, Jason Smith, Xiaofeng Yang, and Alessandro Moschitti. 2008b. BART: A modular toolkit for coreference resolution. \emph{In Proceedings of the ACL-08: HLT Demo Session}, pages 9-12.\\  \\
\big[6\big]Laura Hasler, Constantin Orasan, and Karin Naumann.2006. NPs for events: Experiments in coreference annotation. \emph{In Proceedings of the 5th International Conference on Language Resources and Evaluation,} pages 1167-1172.\\  \\
\big[7\big] Jin-Dong Kim, Sampo Pyysalo,Tomoko Ohta,Robert Bossy, Ngan Nguyen,Jun'ichi Tsujii Overview of BioNLP Shared Task 2011, \emph{In Proceedings of BioNLP Shared Task 2011 Workshop}, pages 1-6 Portland, Oregon, USA, 24 June, 2011.  2011 Association for Computational \\  \\
\big[8\big]   Ronen Feldman and James Sanger.The Text Mining Handbook:  Advanced Approaches in Analyzing Unstructured Data.   Cambridge University Press, Cam-bridge, MA, USA, December 2006.\\  \\
\big[9\big]  Fabio Celli and Massimo Poesio. Improving Relation Extraction with Anaphora In Italian DAARC 2011\\  \\
\big[10\big] M.Novak. Utilization of Anaphora in Machine Translation in \emph{WDS 2011 Proceedings of Contributed Papers: Part I, ser}. WDS 11, 2011, pages. 155 - 160.WDS'11 Proceedings of Contributed Papers, Part I, 155-160, 2011 \\  \\
\big[11\big] Christian Hardmeier and Marcello Federico. 2010. Modeling Pronominal Anaphora in Statistical Machine Translation. \emph{In International Workshop on Spoken Language Translation (IWSLT)}, Paris, December 2nd and 3rd, 2010, pages 283 -289\\  \\
\big[12\big] Ruslan Mitkov ,Sung-Kwon Choi and Randall Sharp. ANAPHORA RESOLUTION IN MACHINE TRANSLATION in \emph{Proceedings of the Sixth International Conference on Theoretical and Methodological Issues in Machine Translation}, pages 87-98 \\  \\
\big[13\big] Liddy, E.D. Natural Language Processing. \emph{In Encyclopedia of Library and Information Science, 2nd} Ed. NY. Marcel Decker, Inc. 2001\\  \\
\big[14\big] Gobinda G Chowdhury. Natural language processing,2003. \emph{In Annual review of information science and technology} vol 37, pages 51-89\\  \\
\big[15\big]  Sætre, Rune, Kazuhiro Yoshida, Akane Yakushiji, Yusuke Miyao, Yuichiro Matsubayashi and Tomoko Ohta. AKANE System: Protein-Protein Interaction Pairs in BioCreAtIvE2 Challenge, PPI-IPS subtask. \emph{In Proceedings of the Second BioCreative Challenge Evaluation Workshop}, pages 209-212, April 2007. CNIO. \\  \\
\big[16\big] Joseph Olive, Caitlin Christianson, and John McCary. 2011.\emph{Handbook of Natural Language Processing and Machine Translation: DARPA Global Autonomous Language Exploitation}. Springer Publishing Company, Inc., 1st edition.(page 97) \\  \\
\big[17\big] Huihsin Tseng, Pichuan Chang, Galen Andrew, Daniel Jurafsky and  Christopher Manning. A conditional random field word segmenter for sighan bakeoff 2005,  \emph{In Proceedings of the Fourth SIGHAN Workshop on Chinese Language Processing}, pages 168-171, Jeju Island, Korea.\\  \\
\big[18\big] Pi-Chuan Chang, Michel Galley, and Christopher D. Manning. Optimizing Chinese Word Segmentation for Machine Translation Performance 2008  \emph{in Proceedings of the Third Workshop on Statistical Machine Translation}, pages 224-232 \\  \\
\big[19\big] Hai Zhao, Chang-Ning Huang and Mu Li An. Improved Chinese Word Segmentation System with Conditional Random Field  in \emph{Proceedings of the Fifth SIGHAN Workshop on Chinese Language Processing} 2006, pages 162-165, Sydney.\\ \\
\big[20\big] Manabu Sassano. An empirical study of active learning with support vector machines for japanese word segmentation. \emph{In Proceedings of the 40th Annual Meeting of the Association for ComputationalbLinguistics}, pages 505-512, 2002 \\ \\
\big[21\big] Josef Steinberger , Massimo Poesio Mijail A. Kabadjov and  Karel Jezek. Two uses of anaphora resolution in summarization in \emph{ J. Steinberger et al.Information Processing and Management} 43 (2007), pages 1663-1680 \\  \\
\big[22\big] Ruslan Mitkov, Richard Evans, Constantin Orasan, Le An Ha, and Viktor Pekar. Anaphora Resolution: To What Extent Does It Help NLP Applications?  DAARC 2007, LNAI 4410, pages. 179-190, 2007\\  \\
\big[23\big] Heeyoung Lee, Angel Chang, Yves Peirsman, Nathanael Chambers, Mihai Surdeanu, and Dan Jurafsky (2013). Deterministic coreference resolution based on entity-centric, precision-ranked rules. \emph{Computational Linguistics}, pages 1-54, January 2013. 
\big[24\big] Kees van Deemter and Rodger Kibble. 2000. On coreferring: Coreference in MUC and related annotation schemes. \emph{Computational Linguistics,} 26(4):629-637. Squib \\  \\
\big[25\big] Marco Maggini, Natural Language Processing Part 3: Syntax  grammar chunking  constituents   \\  \\
\big[26\big] Jirka Hana. Intro to Linguistics - Syntax 1, November 7, 2011 \\  \\
\big[27\big] Jean Eggenschwiler and Emily Dotson. Cliffs Quick Review Writing: Grammar, Usage, and Style Paperback - May 29, 2001 by Jean Eggenschwiler  (Author), Emily Dotson Biggs (Author) \\  \\
\big[28\big] Andrew Radford. An Introduction to English Sentence Structure, Cambridge, 2004, ISBN-13 978-0-511-50666-6 \\  \\
\big[29\big] Andrew Carnie. Syntax,Oxford (2001)\\  \\
\big[30\big] Adrian Brasoveanu Anaphora Resolution Spring 2010, UCSC \\  \\
\big[31\big] Massimo Poesio, Simone Ponzetto, and Yannick Versley. 2011. Computational models of anaphora resolution:A survey. Unpublished. \\  \\
\big[32\big] Xiaofeng Yang, Jian Su, Jun Lang, Chew Lim Tan, and Sheng Li. 2008a. An entity-mention model for coreference resolution with inductive logic programming. \emph{In Proceedings of ACL-08: HLT}, pages 843-851.\\  \\
\big[33\big] Yimeng Zhang and  Yangbo Zhu. Machine Learning for Coreference Resolution: Recent Developments \\  \\
\big[34\big] Shumin Wu and Nicolas Nicolov. Coreference Resolution, A Machine Learning Approach  \\  \\
\big[35\big] Pradheep Elango. Coreference Resolution: A Survey,Technical Report, University of Wisconsin Madison.  \\  \\
\big[36\big] Hannaneh Hajishirzi Leila Zilles Daniel S. Weld Luke Zettlemoyer. Joint Coreference Resolution and Named-Entity Linking with Multi-pass Sieves.   \emph{In proceeding Conference on Empirical Methods on Natural Language Processing}, pages 289-299. ACL. \\  \\
\big[37\big] Hoifung Poon and Pedro Domingos. 2008. Joint unsupervised coreference resolution with Markov logic. \emph{In Proceedings of the 2008 Conference on Empirical Methods in Natural Language Processing}, pages 649-658, Honolulu, HI. ACL. \\  \\
\big[38\big] Shasha Liao and Ralph Grishman. Large Corpus-based Semantic Feature Extraction for Pronoun Coreference. \emph{ In Proceedings 23rd International Conference on Computational}  pages 60-68 \\  \\
\big[39\big] Ruslan Mitkov. 1999. Anaphora resolution: The state of the art. Technical Report (Based on the COLING/ACL-98 tutorial on anaphora resolution), University of Wolverhampton, Wolverhampton. \\  \\
\big[40\big] Fredrik Olsson. A Survey of Machine Learning for Reference Resolution in Textual Discourse. Technical report, Swedish Institute of Computer Science, Kista, 2004.\\  \\
\big[41\big] Aria Haghighi and Dan Klein. 2009. Simple Coreference Resolution with Rich Syntactic and Semantic Features. In Proceedings of the 2009 Conference on Empirical Conference in Natural Language Processing.\\  \\
\big[42\big] Delip Rao and Paul McNamee and Mark Dredze. Streaming cross document entity coreference resolution. \emph{In: Conference on Computational Linguistics (COLING) (2010)} \\  \\
\big[43\big] Rahman Altaf and Vincent Ng. 2009.Supervised models for coreference resolution. \emph{ In Proceedings of the 2009 Conference on Empirical Methods in Natural Language Processing (EMNLP)}, pages 968-977, Suntec. Vincent Ng \\  \\
\big[44\big] David Bean and Ellen Riloff. 2004. Unsupervised learning of contextual role knowledge for coreference resolution. \emph{ In Proceedings  of HLT/NAACL}, pages 297-304. \\  \\
\big[45\big] Unsupervised Models for Coreference Resolution, Vincent Ng \\  \\
\big[46\big] Hobbs, Jerry R., 1976. Pronoun Resolution.  Research Report 76-1, Department of Computer Sciences, City College, City University of New York. August 1976.  \\  \\
\big[47\big] Hobbs, Jerry R., 1978, "Resolving Pronoun References", Lingua, Vol.  44, pp. 311-338.  Also in Readings in Natural Language Processing, B. Grosz, K. Sparck-Jones, and B. Webber, editors, pp. 339-352, Morgan Kaufmann Publishers, Los Altos, California. (a shorter version of the original) \\  \\
\big[48\big] Grosz, Barbara. J., Aravind. K. Joshi, and Scott Weinstein. 1995. Centering: A framework for modeling the local coherence of discourse. \emph{Computational Linguistics}, 21(2):202-225. (The paper originally appeared as an unpublished manuscript in 1986.). \\  \\
\big[49\big] Gasperin Statistical anaphora resolution Caroline Gasperin and Ted Briscoe \emph{in Proceedings of the 22nd International Conference on Computational Linguistics (Coling 2008)} pages 257-264, Manchester , August 2008 \\  \\ 
\big[50\big] http://www.nactem.ac.uk/enju/ Enju parser \\  \\
\big[51\big] Yusuke Miyao and Jun'ichi Tsujii. 2002. Maximum Entropy Estimation for Feature Forests. \emph{In Proceedings of HLT 2002}.\\  \\
\big[52\big] Yusuke Miyao and Jun'ichi Tsujii. 2003. Probabilistic modeling of argument structures including non-local dependencies. \emph{In Proceedings of the Conference on Recent Advances in Natural Language Processing (RANLP)} 2003, pages, 285-291 \\  \\
\big[53\big] Yusuke Miyao, Takashi Ninomiya, and Jun'ichi Tsujii. 2004. Corpus-oriented Grammar Development for Acquiring a Head-driven Phrase Structure Grammar from the Penn Treebank. In Proceedings of IJCNLP-04.\\  \\
\big[54\big] Yusuke Miyao and Jun'ichi Tsujii. 2005. Probabilistic Disambiguation Models for Wide-Coverage HPSG Parsing. \emph{In Proceedings of ACL-2005}, pages. 83-90.\\  \\
\big[55\big] http://www.nactem.ac.uk/y-matsu/geniass/ Genia sentence splitter\\  \\
\big[56\big] Takashi Ninomiya, Takuya Matsuzaki, Yoshimasa Tsuruoka, Yusuke Miyao and Jun'ichi Tsujii. 2006. Extremely Lexicalized Models for Accurate and Fast HPSG Parsing. \emph{In Proceedings of EMNLP 2006.}\\  \\
\big[57\big] Takashi Ninomiya, Takuya Matsuzaki, Yusuke Miyao, and Jun'ichi Tsujii. 2007. A log-linear model with an n-gram reference distribution for accurate HPSG parsing. \emph{In Proceedings of IWPT 2007}.\\  \\
\big[58\big] Yusuke Miyao and Jun'ichi Tsujii. 2008. Feature Forest Models for Probabilistic HPSG Parsing. \emph{Computational Linguistics.} 34(1), pages 35-80, MIT Press.\\  \\
\big[59\big] http://www.nactem.ac.uk/enju/demo.html\\  \\
\big[60\big] Sætre, Rune, Kazuhiro Yoshida, Akane Yakushiji, Yusuke Miyao, Yuichiro Matsubayashi and Tomoko Ohta., AKANE System: Protein-Protein Interaction Pairs in BioCreAtIvE2 Challenge, PPI-IPS subtask.\emph{In Proceedings of the Second BioCreative Challenge Evaluation Workshop}. pages. 209-212, April 2007. CNIO. \\  \\
\big[61\big] Yoshimasa Tsuruoka., A simple C++ library for maximum entropy classification, http://www-tsujii.is.s.u-tokyo.ac.jp/~tsuruoka/maxent/ , .\\  \\
\big[62\big] Jin-Dong Kim, Tomoko Ohta, Yuka Tateishi, and Jun'ichi Tsujii. 2003. GENIA corpus - a semantically annotated corpus for bio-textmining. \emph{Bioinformatics, 19(Suppl.1)}:180-182. \\  \\
\big[63\big]http://2011.bionlp-st.org/ bionlp shared task 2011\\  \\
\big[64\big] Miji Choi, Karin Verspoor and Justin Zobel.2014. Evaluation of coreference resolution for biomedical text.\emph{In proceeding of MedIR 2014, page 2.} \\ \\
\big[65\big]  Yannick Versley, Alessandro Moschitti, Massimo Poesio and Xiaofeng Yang.2008. Coreference systems based on kernel methods. \emph{In Proceedings of COLING}, pages 961-968.\\  \\
\big[66\big] Ninomiya, Takashi, Yoshimasa Tsuruoka, Yusuke Miyao, and Jun'ichi Tsujii. 2005. Efficacy of beam thresholding, unification filtering and hybrid parsing in probabilistic HPSG parsing. \emph{In Proceedings of the 9th International Workshop on Parsing Technologies}, pages 103-114, Vancouver \\  \\
\big[67\big] www.tagtog.net \\  \\
\big[68\big] Cejuela,J.M., McQuilton,P., Ponting,L. et al. tagtog: interactive and text-mining-assisted annotation of gene mentions in PLOS full-text articles. Database (2014) Vol. 2014: article ID bau033; doi:10.1093/database/bau033. \\  \\
\big[69\big] Yu hsiang Lin and Tyne Liang. 2004. Pronominal and sortal anaphora resolution for biomedical literature. \emph{In In Proceedings of ROCLING XVI: Conference on Computational Linguistics and Speech Processing}. \\  \\
\big[70\big] Jennifer D'Souza and Vincent Ng. Anaphora Resolution in Biomedical Literature: A Hybrid Approach. \emph{In Proceeding of BCB '12 Proceedings of the ACM Conference on Bioinformatics}, Computational Biology and Biomedicine Pages 113-122 \\  \\
\big[71\big] http://bionlp-st.dbcls.jp/CO/eval-test/\\  \\
\big[72\big] http://bionlp-st.dbcls.jp/CO/eval-development/\\  \\
\big[73\big] Mitkov, Ruslan. 2002. Anaphora Resolution. London: Pearson Longman.\\  \\


