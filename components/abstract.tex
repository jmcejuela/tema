% Abstract for the TUM report document
% Included by MAIN.TEX


\clearemptydoublepage
\phantomsection
\addcontentsline{toc}{chapter}{Abstract}





\vspace*{2cm}
\begin{center}
{\Large \bf Abstract}
\end{center}
\vspace{1cm}
This thesis introduces a new coreference resolution system for biomedical texts. This system identifies and links expressions (definite nouns and pronouns), which refer to proteins. I focus on improving the current state-of-the-art in coreference resolution for abstracts and, a novel aspect, apply the system on full-text articles. The system uses syntactic rules to resolve pronouns, and  a string-matching method  and domain knowledge together with the C5.0 classifier to resolve definite noun anaphoras. \\
  Furthermore, the lack of a corpus with full text articles negatively impacts  this area of research. To overcome this problem, I annotated 10 full-text documents, containing: proteins, coreference, and anaphora-antecedent relations.
  During this thesis I collected statistics from the existing BioNLP corpus of abstracts and reviewed current coreference resolution methods in other non-biomedical corpora.
  The system achieves 73\% precision and 75\% recall in the BioNLP development set and 72\% precision and 57\% recall in the test set. On my annotated full-text articles, the precision is 75\% and recall is 65\%.

% Juanmi: it makes no sense to highlight the creation of the full text corpus (btw 8 or 10 documents?) here in the abstract and then on the actual thesis body don't even dedicate to this aspect a small section. That is, either you remove this comment from the abstract, or rather create a section in the body (better)

% Juanmi: highlight that your performance is better than the current state of the art on abstracts.